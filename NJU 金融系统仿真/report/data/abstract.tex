\begin{abstract}

贷款审批是普通商业银行的一项重大职能环节,在银行风险管理中占据着极其重要的角色。银行通过对借款者的信
用进行评估,便可以更加准确的判断是否批准贷款,以及以多高的利率、多大的款额借给需求方,实现收益管理。
因此,对于银行来说,判断借款者未来是否会违约以及违约大小便成了一个非常重要的问题。\par

本文通过互联网上两组贷款者申请贷款的历史信息记录及其最终违约与否的数据集,结合逻辑回归、分类树和随机森林这
三种方法对借款者未来违约的概率进行预测,并通过准确率,KS统计量,AUC等对三种模型的有效性进行了对比。结果显示随机
森林有效性最高,其次是逻辑回归,最差是分类树。\par

\end{abstract}

\keywords{信用评估;信用风险;逻辑回归;分类树;随机森林}
