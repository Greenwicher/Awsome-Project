%%%%%%%%%%%%%%%%%%%%%%%%%%%%%%%%%%%%%%%%%
% Structured General Purpose Assignment
% LaTeX Template
%
% This template has been downloaded from:
% http://www.latextemplates.com
%
% Original author:
% Ted Pavlic (http://www.tedpavlic.com)
%
% Note:
% The \lipsum[#] commands throughout this template generate dummy text
% to fill the template out. These commands should all be removed when 
% writing assignment content.
%
%%%%%%%%%%%%%%%%%%%%%%%%%%%%%%%%%%%%%%%%%

%----------------------------------------------------------------------------------------
%	PACKAGES AND OTHER DOCUMENT CONFIGURATIONS
%----------------------------------------------------------------------------------------

\documentclass{article}

\usepackage{fancyhdr} % Required for custom headers
\usepackage{lastpage} % Required to determine the last page for the footer
\usepackage{extramarks} % Required for headers and footers
\usepackage{graphicx} % Required to insert images
\usepackage{lipsum} % Used for inserting dummy 'Lorem ipsum' text into the template
\usepackage{authblk}
\usepackage{enumerate}
\usepackage{booktabs}
\usepackage{amsmath, amssymb}
\usepackage{multirow}
\usepackage{color}

% Margins
\topmargin=-0.45in
\evensidemargin=0in
\oddsidemargin=0in
\textwidth=6.5in
\textheight=9.0in
\headsep=0.25in 

\linespread{1.1} % Line spacing

% Set up the header and footer
\pagestyle{fancy}
\lhead{\hmwkAuthorName} % Top left header
\chead{\hmwkClassID\ (\hmwkClassInstructor)} % Top center header
\rhead{\hmwkTitle} % Top right header
\lfoot{\lastxmark} % Bottom left footer
\cfoot{} % Bottom center footer
\rfoot{Page\ \thepage\ of\ \pageref{LastPage}} % Bottom right footer
\renewcommand\headrulewidth{0.4pt} % Size of the header rule
\renewcommand\footrulewidth{0.4pt} % Size of the footer rule

\setlength\parindent{0pt} % Removes all indentation from paragraphs

   
%----------------------------------------------------------------------------------------
%	NAME AND CLASS SECTION
%----------------------------------------------------------------------------------------

\newcommand{\hmwkTitle}{Group Project Plan} % Assignment title
\newcommand{\hmwkClassID}{IE5504} % Course/class
\newcommand{\hmwkClassName}{Systems Modeling and Advanced Simulation} % Course/class
\newcommand{\hmwkClassInstructor}{Prof. Lee Loo Hay} % Teacher/lecturer
\newcommand{\hmwkAuthorName}{LIU Weizhi} % Your name
\newcommand{\hmwkAuthorEmail}{weizhiliu2009@gmail.com} % Your email

%----------------------------------------------------------------------------------------
%	TITLE PAGE
%----------------------------------------------------------------------------------------

\title{
\vspace{2in}
\textmd{\textbf{\hmwkClassID \ - \hmwkClassName}}\\
\textmd{\textbf{\hmwkTitle}}\\
\vspace{0.1in}\large{\textit{supervisor: \hmwkClassInstructor}}\\
\vspace{1in}
\textmd{\textbf{Group Member:} \\ \vspace{0.2in} Li Qingxuan, Li Yue, Liu Weizhi,\\ Wong Manyu, Yuan yuhe}
\vspace{1.5in}
}

\author{\textbf{\hmwkAuthorName} \thanks{\hmwkAuthorEmail}}
\affil{Department of Industrial \& Systems Engineering \\ National University of Singapore}
\date{\today} % Insert date here if you want it to appear below your name

%----------------------------------------------------------------------------------------

\begin{document}

\maketitle

%----------------------------------------------------------------------------------------
%	TABLE OF CONTENTS
%----------------------------------------------------------------------------------------

%\setcounter{tocdepth}{1} % Uncomment this line if you don't want subsections listed in the ToC

% \newpage
% \tableofcontents
% \newpage

\newpage

\section{Introduction}
We intend to investigate the multi-armed bandit problem which can be formulated as below: \\
\textbf{Maximize your total reward $R$ with at most $n$ trials given $K$ slot machines (bandit) whose reward distribution is quite different.}\\
\\
This problem is very similiar to what OCBA are supposed to solve. Treat the slot machines as possible designs and the amount of tries as computing resources. \textbf{Consequently, we are very interesed in comparing the performance of OCBA and traditional methods to solve multi-armed bandit problem and figure out the reason why they might perform differently.}\\
\\
To ease the problem without loss of generality for multi-armed bandit problem, we assume the reward of slot machine $k$ obeys Bernoulli distribution $p_{k}$. Therefore, the conjugate distribution is Beta distribution.
\section{Strategy}
We propose two preliminary approaches to solve the problem described as section 1. More detailed algorithms will be developed further. 
\subsection{Bayesian Bandit}
The process of Bayesian bandit algorithm is as follow:
\begin{enumerate}
  \item Initialize the Bernoulli success rate for all bandits and generate a success/failure (1/0) sequence with length $n$ for each bandit. (Pseudo Random)
  \item Calculate the APCS (Approximate Probability of Correct Selection \footnote{please refer to \textbf{Stochastic Simulation Optimization - An Optimal Computing Budget Allocation}, P37} or Stochastic Ordering) of all bandits based on their prior distribution.
  \item Find the bandit B with highest APCS (if not unique, then uniformly randomly choose a bandit)
  \item Observe the reward of bandit B from the given sequence
  \item Update the posterior distribution of bandit B
  \item  Stop if total number of trials run out otherwise return to step 2
\end{enumerate}
\subsection{Optimal Computing Budget Allocation}
The process of Optimal Computing Budget Allocation is as follow:
\begin{enumerate}
  \item Initialize the Bernoulli success rate for all bandits and generate a success/failure (1/0) sequence with length $n$ for each bandit.
  \item For each bandit, initially try $N_{0}$ times and calculate their corresponding mean and variance
  \item Find the bandit B with highest mean as the best bandit in the initial period.
  \item Allocate the number of trials to different bandit according to the allocation rule \footnote{please refer to the allocation rule at \textbf{Stochastic Simulation Optimization - An Optimal Computing Budget Allocation}, P46} in order to maximize the probability of correct selection.
  \item Try the bandit for respective allocated nums and then calculated the mean and variance for each bandit.
  \item Stop if total number of trials run out otherwise return to step 3
\end{enumerate}
\section{Evaluation}
We adopt the Pseudo Regret as a performance measure for the algorithm to find the optimal bandit.
$$ Pseudo \ Regret = \max_{i} \sum_{t=1}^{n} R_{i,t} - \sum_{t=1}^{n} R_{I_{t}, t}$$ where $I_{t}$ is the bandit selected at time $t$.
\section{Extension}
If time is enough, some general scenarios could be considered
\begin{itemize}
  \item Find more strategies to solve the multi-armed bandit problem and make comparisions.
  \item Integration of markov decision process to solve multi-armed bandit problem.
  \item Modify the reward distribution to more relastic distributions.
\end{itemize}
\end{document}
